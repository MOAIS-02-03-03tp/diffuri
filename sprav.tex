\documentclass[12pt]{article}
\usepackage[a4paper, total={7in,10in}]{geometry}
\usepackage{polyglossia}
\usepackage{ragged2e}
\usepackage{amsmath}
\usepackage{amssymb}
\usepackage{microtype}
\usepackage{graphicx}
\let\ORIincludegraphics\includegraphics
\renewcommand{\includegraphics}[2][]{\ORIincludegraphics[scale=0.65,#1]{#2}}
\usepackage{changepage}
\usepackage{hyperref}
\usepackage{cancel}
\graphicspath{{./images/}}
\setmainlanguage{russian}
\setotherlanguage{english}
\newfontfamily\russianfont[Script=Cyrillic]{Times New Roman}
\newfontfamily\englishfont{Times New Roman}
\setlength{\parindent}{0em}
\setlength{\parskip}{6pt}

\def\posl#1#2{\{#1_{#2}\}}
\DeclareMathOperator*{\sh-like}{\sinh-like}
\DeclareMathOperator*{\ch-like}{\cosh-like}
\DeclareMathOperator*{\th-like}{\tanh-like}
\DeclareMathOperator*{\cth-like}{\coth-like}
\DeclareMathOperator*{\tg-like}{\tan-like}
\DeclareMathOperator*{\ctg-like}{\cot-like}
\DeclareMathOperator*{\arctg-like}{\arctan-like}
\DeclareMathOperator*{\arcctg-like}{\arctan-like}

\begin{document}
  \tableofcontents
  \pagebreak
  \section{Введение в теорию обыкновенных дифференциальных уравнений(ТОДУ)}
  \subsection*{Предмет и задачи курса ДУ}
  \subsection*{Обыкновенные дифференциальные уравнения}
  \begin{enumerate}
    \item Классификация и методы получение аналитических решений ОДУ, в том числе и в CAE системах(Mapple,MatLab)
    \item Элементы качественного анализа
    \item Основные численные методы
  \end{enumerate}
  ТОДУ позволяет изучать всевозможные эволюционные процессы: обладающие свойствами:
  \begin{enumerate}
    \item Детерминированности (состояние в прошлом и будущий ход событий однозначно определяются
    состоянием системы в настоящем)
    \item Конечномерности(число параметров необходимых для описания поведения системы конечно)
    \item Дифференцируемости(фазовое пространстро имеет структуру дифференцируемого многообразия,
    а его изменение описывается дифференцируемыми функциями)
  \end{enumerate}
  \subsection*{Объект изучения - обыкновенные дифференциальные уравнения n-порядка}
  $F(x,y,y',\dots,y^{(n)})=0$\\
  $x \in I \sqsubset R,y(x)$ - подлежит определению, n-порядок ДУ\\
  \[y^{(n)}=f(x,y,y',\dots,y^{(n-1)})\] \\
  ОДУ n-порядка разрешенно относительно производной (n)-го порядка
  \subsection*{ОДУ первого порядка выраженное относительно производной}
  \[y'=f(x,y),D,y(x),I\]\\
  D-область определения f(x,y)($D \sqsubseteq R^2),x \in I, I \sqsubseteq R$\\
  Замечание(часто в ТОДУ)$\overset{.}{x}=\varphi(t,x),D,x(t),I$\\
  \underline{Определение: } Пусть f(x,y) действительнозначная функция действительных переменных x и y
  c областью определения $D \sqsubseteq R^2$. функция y(x)($x \in I, I\sqsubseteq R$)
  для которой всюду на I выполняется равенство y'=f(x,y) называется решением дифференциального уравнения
  (1). График решения называется интегральной кривой.\\
  Пример 1: y'=y-x,$R^2,1+x+C \; e^x,R$\\
  \begin{gather*}
    y'=y-x,y(x)-? \; x-\text{парм.}\\
    \text{Ввд } z(x)=y(x)-x\\
    y'=\frac{dy}{dx}\\
    x'=y'-1\\
    z'+1=z\\
    z'=z-1\\
    \frac{dz}{dx}=(z-1)\\
    \int \frac{dz}{z-1}=\int dx => C+ln(z-1)=x+C\\
    ln|z-1|=x+c\\
    |z-1|=e^{x+C}\\
    z-1=e^x\\
    z=Ce^x+1\\
    y(x)-x=Ce^x+1\\
    y(x)=x+1+Ce^x
  \end{gather*}
\end{document}